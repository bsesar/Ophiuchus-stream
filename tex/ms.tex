%\documentclass[12pt,preprint]{aastex}
\documentclass[iop]{emulateapj}
\usepackage{amsmath}

\begin{document}

\title{The Nature and Orbit of the Ophiuchus Stream}

\author{Branimir Sesar\altaffilmark{1,2}}
\author{Jo Bovy\altaffilmark{3,4}}
\author{Edouard J.~Bernard\altaffilmark{5}}
\author{Nelson Caldwell\altaffilmark{6}}
\author{Judith G.~Cohen\altaffilmark{7}}
\author{Morgan Fouesneau\altaffilmark{1}}
\author{Christian I.~Johnson\altaffilmark{6}}
\author{Melissa Ness\altaffilmark{1}}
%\author{Maria Bergemann\altaffilmark{1}}
\author{Annette M.~N.~Ferguson\altaffilmark{5}}
\author{Nicolas F.~Martin\altaffilmark{8,1}}
\author{Hans-Walter Rix\altaffilmark{1}}
\author{Edward F.~Schlafly\altaffilmark{1}}
%\author{builders}
\author{W.~S.~Burgett\altaffilmark{15}}
\author{K.~C.~Chambers\altaffilmark{9}} 
%\author{L.~Denneau\altaffilmark{9}} 
%\author{P.~Draper\altaffilmark{10}} 
%\author{H.~Flewelling\altaffilmark{9}} 
%\author{T.~Grav\altaffilmark{11}} 
%\author{J.~N.~Heasley\altaffilmark{9}} 
\author{K.~W.~Hodapp\altaffilmark{9}} 
%\author{M.~E.~Huber\altaffilmark{9}} 
%\author{R.~Jedicke\altaffilmark{9}} 
\author{N.~Kaiser\altaffilmark{9}} 
%\author{R.-P.~Kudritzki\altaffilmark{9}}
%\author{G.~A.~Luppino\altaffilmark{9}}
%\author{R.~H.~Lupton\altaffilmark{12}}
\author{E.~A.~Magnier\altaffilmark{9}}
%\author{N.~Metcalfe\altaffilmark{10}}
%\author{D.~G.~Monet\altaffilmark{13}}
%\author{J.~S.~Morgan\altaffilmark{9}}
%\author{P.~M.~Onaka\altaffilmark{9}}
%\author{P.~A.~Price\altaffilmark{12}}
%\author{C.~W.~Stubbs\altaffilmark{14}}
%\author{W.~Sweeney\altaffilmark{9}}
\author{J.~L.~Tonry\altaffilmark{9}}
%\author{R.~J.~Wainscoat\altaffilmark{9}}
\author{C.~Waters\altaffilmark{9}}

\altaffiltext{1}{Max Planck Institute for Astronomy, K\"{o}nigstuhl 17, D-69117
                 Heidelberg, Germany}
\altaffiltext{2}{Corresponding author: bsesar@mpia.de}
\altaffiltext{3}{Institute for Advanced Study, Einstein Drive, Princeton, NJ
                 08540, USA}
\altaffiltext{4}{John Bahcall Fellow}
\altaffiltext{5}{SUPA, Institute for Astronomy, University of Edinburgh, Royal
                 Observatory, Blackford Hill, Edinburgh EH9 3HJ, UK}
\altaffiltext{6}{Harvard-Smithsonian Center for Astrophysics, 60 Garden Street,
                 Cambridge, MA 02138, USA}
\altaffiltext{7}{Division of Physics, Mathematics and Astronomy, California
                 Institute of Technology, Pasadena, CA 91125,
                 USA}
\altaffiltext{8}{Observatoire astronomique de Strasbourg, Universit\'e de
                 Strasbourg, CNRS, UMR 7550, 11 rue de l'Universit\'e, F-67000
                 Strasbourg, France}
\altaffiltext{9}{Institute for Astronomy, University of Hawaii at Manoa, Honolulu, HI 96822, USA}
\altaffiltext{10}{Department of Physics, Durham University, South Road, Durham DH1 3LE, UK}

\altaffiltext{11}{Department of Physics and Astronomy, Johns Hopkins University, 3400 North Charles Street, Baltimore, MD 21218, USA}

\altaffiltext{12}{Department of Astrophysical Sciences, Princeton University, Princeton, NJ 08544, USA}

\altaffiltext{13}{US Naval Observatory, Flagstaff Station, Flagstaff, AZ 86001, USA}

\altaffiltext{14}{Department of Physics, Harvard University, Cambridge, MA 02138, USA}

\altaffiltext{15}{GMTO Corporation, 251 S.~Lake Ave., Suite 300, Pasadena, CA
                  91101, USA } 

\begin{abstract}
The Ophiuchus stream is the most recently discovered stellar tidal stream in the
Milky Way \citep{ber14b}. We present high-quality spectroscopic data for 14
stream member stars obtained using Keck and MMT telescopes. We confirm the
stream as a fast moving ($v_{los}\sim290$ km s$^{-1}$), kinematically-cold group
($\sigma_{v_{los}}\lesssim1$ km s$^{-1}$) of $\alpha-$enhanced and metal-poor
stars (${\rm [\alpha/Fe]\sim0.4}$ dex, ${\rm [Fe/H]\sim-2.0}$ dex). Using a
probabilistic technique, we model the stream simultaneously in line-of-sight
velocity, color-magnitude, coordinate, and proper motion space. We find that
that the stream extends from 8 to 9.5 kpc from the Sun and that its deprojected
length is $\sim1.6$ kpc. The analysis of the stellar population contained in the
stream suggests that it is $\sim13$ Gyr old, and that its initial stellar mass
was $\sim2\times10^4$ $M_\sun$ (or at least $\ga4\times10^3$ $M_\sun$). We do
not detect a significant overdensity of stars along the stream that would
indicate the presence of a progenitor, and conclude that the stream is all that
is left of the progenitor. Assuming a potential for the Milky Way that is
consistent with a large variety of dynamical constraints, we fit the orbit of
the stream. We find that the stream has an orbital period of $\sim360$ Myr, and
is on a fairly eccentric orbit ($e\sim0.68$) with a pericenter of $\sim3.5$ kpc
and an apocenter of $\sim17.5$ kpc. Based on the available information, we
conclude that the progenitor of the stream was a globular cluster and estimate
the time of disruption at $\sim250$ Myr ago.
\end{abstract}

\keywords{globular clusters: general --- Galaxy: halo --- Galaxy: kinematics and dynamics --- Galaxy: structure}

\section{Introduction}\label{introduction}

One of the main goals of Galactic astronomy is the measurement of the Milky
Way's gravitational potential, because knowledge of it is required in any study
of the dynamics or evolution of the Galaxy. An important tool in this
undertaking are stellar tidal streams, remnants of accreted Milky Way satellites
that were disrupted by tidal forces and stretched into filaments as they orbited
in the Galaxy's potential. The orbit of a stream is sensitive to the properties
of the potential and thus can be used to constrain the potential over the range
of distances spanned by the stream (e.g., \citealt{kop10, new10, ses13, bel14}).
In this context, the recently discovered Ophiuchus stellar stream \citep{ber14b}
is very interesting because it is located fairly close to the Galactic center
(galactocentric distance of $\sim5$ kpc), and as such probes the part of the
potential that other known stellar tidal streams do not probe.

The Ophiuchus stream is a $\sim2.5\arcdeg$ long and $7\arcmin$ wide stellar
stream that was recently discovered by \citet{ber14b} in the Pan-STARRS1
photometric catalog (PS1; \citealt{kai10}). Bernard et al.~inferred from its
color-magnitude diagram that it is consistent with an old ($\ga10$ Gyr) and
relatively metal-poor population ($[Fe/H]\sim -1.3$ dex) located $\sim9$ kpc
away at $(l, b)\sim(5\arcdeg, +32\arcdeg)$. They did not detect a progenitor (or
a remnant of it), but suggested that the progenitor would most likely be a
globular cluster.

Due to the lack of proper motion and line-of-sight velocity measurements,
Bernard et al.~could not determine the orbit of stream and thus could not use it
to constrain the potential. Furthermore, without knowing the orbit of the
stream, they could not fully explain two interesting properties of the
Ophiuchus stream, namely, its very short length and the lack of a visible
progenitor. The projected angular length of $2.5\arcdeg$ at a distance of
$\sim9$ kpc implies a projected physical length of $\sim400$ pc for the
Ophiuchus stream. Such as short length suggests that its progenitor must have
been disrupted fairly recently. However, if that was the case, the progenitor
should still be visible as an overdensity of stars somewhere along the stream.
Yet, no progenitor has been detected so far.

To address the above questions, we need to know the orbit of the Ophiuchus
stream, and to determine its orbit we need to measure the stream's line-of-sight
velocity, distance, and proper motion. In Section~\ref{data}, we describe the
data we use in this work; the PS1 photometry and astrometry, the spectroscopic
follow-up of candidate stream members, and the measurement of their
line-of-sight velocities, chemical abundances, and proper motions. In
Section~\ref{results}, we provide a detailed characterization of the stream in
position, velocity, and abundance (7-D) phase space. The constraints obtained in
Section~\ref{results} are then used to constrain and examine the orbit of the
stream (Section~\ref{orbit}). In Section~\ref{conclusions}, we discuss the
nature of the stream's peculiar orbit, highlight the solved and uncovered
puzzles related to the stream, and finally present our conclusions.

\section{Data}\label{data}

\subsection{Overview of the Pan-STARRS1 survey}\label{PS1}

The PS1 survey has observed the entire sky north of declination $-30\arcdeg$ in
five filters covering $400-1000$ nm \citep{stu10, ton12}. The 1.8-m PS1
telescope has a 7 deg$^2$ field of view outfitted with a billion-pixel camera
\citep{hod04, ona08, to09}. In single-epoch images, the telescope can detect
point sources at a signal-to-noise ratio of 5 at 22.0, 22.0, 21.9, 21.0, and
19.8 mag in PS1 $grizy_{P1}$ bands, respectively. The survey pipeline
automatically processes images and performs photometry and astrometry on
detected sources \citep{mag06, mag07}. The uncertainty in photometric
calibration of the survey is $\la0.01$ mag \citep{sch12}, and the astrometric
precision of single-epoch detections is 10 milliarcsec
\citep[hereafter mas]{mag08}.

\subsection{Line-of-sight velocities}

Based on the findings of \citet{ber14b}, we have used the dereddened and shifted
fiducial of the old globular cluster NGC 5904 (from \citealt{ber14a}) to select
$\sim170$ candidate Ophiuchus stream members from the PS1 photometric catalog.
The candidates were observed using the DEIMOS spectrograph on Keck II
\citep{fab02} and using the Hectochelle fiber spectrograph on MMT \citep{sze11}
over a course of two nights.

Seven candidate blue horizontal branch stars were observed with DEIMOS on 2014
May 29$^{\rm th}$ (project ID 2014A-C171D, PI: J.~Cohen). The observations were
made using the $0.8\arcsec$ slit and the high resolution (1200G) grating,
delivering a resolution of 1.2 {\AA} in the 6250-8900 {\AA} range. The spectra
were extracted and calibrated using standard
IRAF\footnote{\url{http://iraf.noao.edu/}} tasks. The uncertainty in the
zero-point of wavelength calibration (measured using sky lines) was
$\lesssim0.04$ {\AA} ($\lesssim2$ km s$^{-1}$ at 6563 \AA).

The line-of-sight velocities of stars observed by DEIMOS were measured by
fitting observed spectra with synthetic template spectra selected from the
\citet{mun05} spectral
library\footnote{\url{http://archives.pd.astro.it/2500-10500/}}. Prior to
fitting, the synthetic spectra were resampled to the same \AA~per pixel scale
as the observed spectrum and convolved with an appropriate Line Spread Function.
The velocity obtained from the best-fit template was corrected to the
barycentric system and adopted as the line-of-sight velocity, $v_{\rm los}$. We
added in quadrature the uncertainty in the zero-point of wavelength calibration
(2 km s$^{-1}$ at 6563 \AA) to the velocity error from fitting.

The remaining 163 Ophiuchus stream candidates were observed with Hectochelle on
2014 June 6$^{\rm th}$ (proposal ID 2014B-SAO-4, PI: C.~Johnson). Observations
were made using the RV31 radial velocity filter, which includes Mg I/Mgb
features in the 5150-5300 {\AA} range. To improve the signal-to-noise ratio
(SNR) of faint targets, we binned the detector by 3 pixels in the spectral
direction, resulting in an effective resolution of $R\sim38,000$.

Hectochelle spectra were extracted and calibrated following \citet{cal09}. To
account for variations in the fiber throughput, the spectra were normalized
before sky subtraction. The normalization factor was estimated using the
strength of several night sky emission lines in the appropriate order. Sky
subtraction was performed using the average of 20-30 sky fibers, using the
method  devised by \citet{kop11}. A comparison of observed and laboratory
positions of sky emission lines did not reveal any significant offsets in
wavelength calibration (i.e., no offsets greater than 0.5 km s$^{-1}$ at 5225
\AA).

The line-of-sight velocities of stars observed by Hectochelle were measured
using the RVSAO package \citep{km98}, by cross-correlating observed spectra with
a synthetic spectrum of an A-type and a G-type giant star (constructed by
\citealt{lat02}). The velocity obtained from the best fitting template was
adopted. To the uncertainty in $v_{\rm los}$, measured by RVSAO, we added (in
quadrature) the uncertainty in the zero-point of wavelength calibration, which
we measured using sky emission lines to be 0.5 km s$^{-1}$. Finally, the
measured velocities were corrected to the barycentric system using the BCVCORR
task.

A comparison of velocities measured from DEIMOS and Hectochelle spectra for star
``bhb6'' (Table~\ref{table1}), shows that the two velocity sets are consistent
within stated uncertainties.

\subsection{Chemical abundances}

Even though the primary goal of spectroscopic observations was to obtain precise
radial velocities, the wavelength range and the resolution of Hectochelle
spectra are sufficient to allow estimates of chemical abundances.

We determined stellar parameters from the continuum-normalised, radial
velocity-corrected spectra using the SMH code of \citet{cas14}, which is built
on the MOOG code of \citet{sne73}. Kurucz model atmospheres \citep{ck04} and a
line list compiled from \citet{fre10} and \citet{yon05} by \citet{cas14} were
used. First, effective temperatures were calculated from dereddened PS1 $g_{P1}$
and $r_{P1}$ bands\footnote{Transformed to the Sloan Digital Sky Survey (SDSS;
\citealt{yor00}) $g$ and $r$ bands using relations of \citet{ton12}.} using
Equation 3 of \citet{ive08} and spectroscopic temperatures were optimized around
this value using the SMH code, by removing abundance trends with line excitation
potential. Parameters of $\log g$ and ${\rm [Fe/H]}$ were determined using 14 Fe
I and 2-3 Fe II lines to achieve ionization balance, and microturbulence was
calculated by removing abundance trends with reduced equivalent width of the
lines. Estimates of the $\alpha-$enhancement were obtained using only the few
available clean Mg, Ca and Ti (I and II) lines, which comprised a total of 6-8
lines per star.

\subsection{Proper motions}

Proper motions are indispensable constraints for the orbit of a stream (e.g.,
\citealt{kop10}) To measure the proper motion of stars in the vicinity of the
Ophiuchus stream, we combine the astrometry provided by USNO-B \citep{mon03} and
2MASS \citep{skr06} catalogs with the PS1 catalog. The USNO-B catalog lists
photometry and astrometry measured from photographic plates in five different
band-passes ($O$, $E$, $J$, $F$, and $N$). The USNO-B plates were exposed at
different epochs, and thus each object in the catalog can have a maximum of five
recorded positions. The 2MASS catalog provides only one position entry per
object.

Since these catalogs are not tied to the same astrometric reference system, we
first calibrate USNO-B and 2MASS positions to the reference frame defined by
positions of galaxies observed in PS1. We define galaxies as objects that have
the difference between Point-Spread-Function (PSF) and aperture magnitudes in
PS1 $r_{P1}$ and $i_{P1}$ bands between 0.3 and 1.0 mag.

The astrometric reference catalog is created by averaging out repeatedly
observed positions of PS1 objects. Between May and June 2012, the region in the
vicinity of the Ophiuchus stream was observed four times in PS1 $g_{P1}$,
$r_{P1}$, and $i_{P1}$ bands. To minimize the uncertainty in astrometry due to
wavelength-dependent effects, such as the differential chromatic refraction
(DCR), we only average out positions observed through the $r_{P1}$-band filter.
Since the astrometric precision of single-epoch detections is 10 mas
\citep{mag08}, the precision of the average position is $\sim5$ mas or better.

The USNO-B astrometry is calibrated following
\citet[see their Section 2.1]{mun04}. First, we calculate the positions
of objects at each of the five USNO-B epochs, using software kindly provided by
J.~Munn. Then, for each USNO-B object we find 100 nearest galaxies, calculate
the median offsets in right ascension and declination between the reference PS1
position and the USNO-B position for these galaxies, and add the offsets to the
USNO-B position in question. This is done separately for each of the five USNO-B
epochs. The single-epoch 2MASS positions are calibrated using the same
procedure.

Having tied the positions for each object at one 2MASS and five USNO-B epochs to
the PS1 astrometric reference frame, we can now check for any additional
systematic uncertainties in the calibrated astrometry. We do so using the
leave-one-out cross-validation. One of the six calibrated positions is witheld, 
and a straight line is fitted to the remaining five positions and the PS1
position. The straight line fit (i.e., essentially a proper motion fit,
neglecting the parallax) is then used to predict the position of an object at
the witheld epoch. The difference between the witheld position and the predicted
position is labeled as $\Delta RA$ or $\Delta Decl$.

\begin{figure}
\plotone{astrometric_residuals.pdf}
\caption{
This plot illustrates the systematic offset in declination ($\Delta Decl$) of
objects observed in the POSS-II Blue epoch of the USNO-B catalog (plate 799), as
a function of the $g_{P1}$-band magnitude. For clarity, only a subset of objects
are plotted. The solid yellow circles show the median $\Delta Decl$ in magnitude
bins, and the solid red circles show the root-mean-square (rms) scatter in
magnitude bins. Note how the brighter objects are systematically offset by
$\sim200$ mas from the fainter objects. The rms scatter indicates that the
systematic precision in this coordinate and epoch is $\sim120$ mas.
\label{usnob_crossvalidation}}
\end{figure}

Inspection of $\Delta RA$ or $\Delta Decl$ values has revealed that the
positions of USNO-B objects depend on magnitude for some epochs (e.g., see
Figure~\ref{usnob_crossvalidation}). We have examined $\Delta RA$ and
$\Delta Decl$ values in different regions of the sky, and have concluded that
these astrometric issues affect individual photographic plates, and are not
specific to a particular photographic bandpass. To remove this dependence, we
subtract a plate-specific and magnitude-dependent offset (e.g., see yellow
circles in Figure~\ref{usnob_crossvalidation}) from original USNO-B positions
{\em before} we calibrate the positions using PS1 galaxies. Since USNO-B does
not provide uncertainty in positions, we adopt the rms scatter of $\Delta RA$
and $\Delta Decl$ values (e.g., see red solid circles in
Figure~\ref{usnob_crossvalidation}) as an estimate of the uncertainty in
position at a given magnitude and epoch.

Finally, to measure the proper motion of an object we fit a straight line to all
the available positions (max.~7). The proper motion of confirmed Ophiuchus
stream members is listed in Table~\ref{table1}.

\section{Characterization of the Ophiuchus stream}\label{results}

\subsection{Line-of-sight velocities\label{vlos}}

\begin{figure}
\plotone{f1.pdf}
\caption{
The distribution of heliocentric line-of-sight velocities of stars observed by
DEIMOS and Hectochelle. The uncertainty in individual $v_{los}$ measurements is
$\lesssim2$ km s$^{-1}$ and the bin size is 25 km s$^{-1}$. The Ophiuchus stream
is detected as a group of stars with $v_{los}\sim290$ km s$^{-1}$.
\label{rv_hist}}
\end{figure}

The $v_{los}$ distribution of stars observed by DEIMOS and Hectochelle is shown
in Figure~\ref{rv_hist}. In this Figure, a group of 14 stars with
$285 < v_{los}/ {\rm km s^{-1}} < 292$ clearly stands out. This group, which we
identify as the Ophiuchus stream, is well-separated from the majority of stars
which have $|v_{los}|<200$ km s$^{-1}$. The positions, velocities, and PS1
photometry of stars in this group are listed in Table~\ref{table1}.

\begin{figure}
\plotone{gl_vs_rv.pdf}
\caption{
Line-of-sight velocities of stars in the Ophiuchus stream are shown as symbols
with error bars. The thick solid line shows the most probable model
($v_{los}(l)=4(l-5)+289.1$ km s$^{-1}$). To illustrate the uncertainty in the
most probable model, the thin semi-transparent red lines show 200 models sampled
from the posterior distribution. The vertical dashed lines show the likely
extent of the stream (see Section~\ref{proper_motion_extent}).
\label{gl_vs_rv}}
\end{figure}

A closer look at $v_{los}$ of stars in the Ophiuchus stream
(Figure~\ref{gl_vs_rv}) suggests that their velocities are changing as a
function of galactic longitude. To fit this possible velocity gradient, we use
an approach similar to the one taken by \citet[see their Section 2.1.1]{mj10}.

We wish to find a set of parameters $\theta$ for which the observations of
stars listed in Table~\ref{table1}, $\mathcal{D}={\bf \{d_k\}}_{1\leq k\leq14}$,
become most likely given the model we describe below. In the current problem,
each data point ${\bf d_k}$ in data set $\mathcal{D}$ is defined by its galactic
longitude and line-of-sight velocity, ${\bf d_k}=\{l_k, v_{los,k}\}$, and their
associated uncertainties. The data points are also considered to be independent.
The likelihood that these data points follow the model defined by the set of
parameters $\theta$, is then
\begin{equation}
    \mathcal{L}(\mathcal{D} | \theta) = \prod_k \mathcal{L}_k({\bf d_k | \theta})\label{total_likelihood},
\end{equation}
where $\mathcal{L}_k({\bf d_k | \theta})$ is the likelihood of data point $k$ to
be generated from the model. Using Bayes Theorem, the probability of a model
given the data, $P(\theta | \mathcal{D})$, is then
\begin{equation}
    P(\theta | \mathcal{D}) \propto \mathcal{L}(\mathcal{D} | \theta)P(\theta)\label{probability_of_a_model},
\end{equation}
where $P(\theta)$ represents our prior knowledge on the model.

To model the velocity as a function of galactic longitude, we use the velocity
gradient $\frac{dv_{los}}{dl}$, the velocity of the stream at galactic longitude
$l=5\arcdeg$, $\overline{v_{los}}$, and the intrinsic velocity dispersion $s$ as
parameters. Given this model, the likelihood of a data point $k$ is defined as
\begin{equation}
    \mathcal{L}_k(l_k, v_{los,k}, \sigma_{v_{los, k}} | \overline{v_{los}}, \frac{dv_{los}}{dl}, s) = \mathcal{N}(v_{los,k} | v(l_k), \sigma^\prime_k),
\end{equation}
where
\begin{equation}
\mathcal{N}(x|\mu,\sigma)=(1/\sqrt{2\pi\sigma^2})\exp(-0.5((x-\mu)^2)/\sigma^2)
\end{equation}
is a normal distribution,
$v(l_k) = \frac{dv_{los}}{dl}(l_k - 5) +\overline{v_{los}}$ is the predicted
velocity at galactic longitude $l_k$, and
$\sigma^\prime_k=\sqrt{s^2 + \sigma^2_{v_{los, k}}}$ is the quadratic sum of the
intrinsic velocity dispersion and the uncertainty in line-of-sight velocity of
data point $k$. The likelihood of all data points can then be calculated using
Equation~\ref{total_likelihood}.

Before we can calculate the probability of a model, we need to define the prior
probabilities of model parameters. As prior probabilities, we adopt priors that
are uniform in these ranges: $270 < \overline{v_{los}}/{\rm km\, s^{-1}} < 320$,
$0 < \frac{dv_{los}}{dl}/{\rm km\, s^{-1}\, deg^{-1}} < 8$,
$0 \leq s/{\rm km\, s^{-1}} < 3$.

To find the most probable model, we calculate
Equation~\ref{probability_of_a_model} over a fine three-dimensional grid in
model parameters, and find values for which $P(\theta | \mathcal{D})$ is
maximal. We find $\frac{dv_{los}}{dl}=4\pm1$ km s$^{-1}$ deg$^{-1}$,
$\overline{v_{los}}=289.1\pm0.5$ km s$^{-1}$, and a very small velocity
dispersion of $s=0.4_{-0.3}^{+0.6}$ km s$^{-1}$. The quoted uncertainties
correspond to central $68\%$ confidence limits (i.e., $1\sigma$ uncertainties).
Thus, we detect a gradient in line-of-sight velocities at a $4\sigma$ level.

\subsection{Chemical abundances\label{abundances}}

The preliminary chemical abundances of five (RGB) stars in the Ophiuchus stream
(that is, the $v_{los}\sim290$ km s$^{-1}$ group), and observed by Hectochelle, 
are listed in Table~\ref{table_abundances}. The uncertainties of the determined
parameters are listed in the notes of Table~\ref{table_abundances}.

\begin{deluxetable}{lrrrrrr}
\tabletypesize{\scriptsize}
\setlength{\tabcolsep}{0.02in}
\tablecolumns{7}
\tablewidth{0pc}
\tablecaption{Chemical abundances of Ophiuchus stream stars\label{table_abundances}}
\tablehead{
\colhead{Name} & \colhead{$T_{eff}$} & \colhead{$\log g$} &
\colhead{${\rm [Fe/H]}$} & \colhead{${\rm [Mg/Fe]}$} &
\colhead{${\rm [Ca/Fe]}$} & \colhead{${\rm [Ti/Fe]}$}  \\
\colhead{ } & \colhead{(K)} & \colhead{(dex)} &
\colhead{(dex)} & \colhead{(dex)} & \colhead{(dex)} &
\colhead{(dex)}
}
\startdata
rgb1 & 5680 & 3.0 & -1.95 & -0.2 & 0.6 & 0.6 \\
rgb2 & 5450 & 2.8 & -2.02 &  0.1 & 0.6 & 0.3 \\
rgb3 & 5700 & 3.0 & -1.95 &  0.2 & 0.3 & 0.5 \\
rgb4 & 5500 & 2.8 & -1.95 &  0.3 & 0.6 & 0.6 \\
rgb5 & 5720 & 3.4 & -1.90 &  0.0 & 0.7 & 0.5 
\enddata
\tablecomments{The uncertainty in $T_{eff}$ is $<200$ K, $<0.4$ dex for
$\log g$, $\lesssim0.2$ dex for ${\rm [Fe/H]}$, and $\sim0.3$ dex for abundances
of $\alpha-$elements.}
\end{deluxetable}

We find the stars in the Ophiuchus stream to be poor in Fe
(${\mathrm [Fe/H]\sim-2.0}$ dex) and enhanced in $\alpha$-elements
(${\mathrm [\alpha/Fe]=0.4\pm0.1}$ dex). Their ${\mathrm [Fe/H]}$ are consistent
within 0.05 dex (root-mean-square scatter), despite fairly large estimated
uncertainties in individual measurements ($\lesssim0.2$ dex). The small scatter
in ${\mathrm [Fe/H]}$ suggests that these stars come from the same single
stellar population.

Based on their position (within $5\arcmin$ of the Ophiuchus stream, as traced
by \citealt{ber14b}), kinematic and chemical properties, we conclude that all
stars listed in Table~\ref{table1} are high-probability members of the Ophiuchus
stream.

\subsection{Color-magnitude diagram}\label{CMD}

The sample of Ophiuchus stream members, which we have identified above using
velocities and metallicities, now gives us an opportunity to further constrain
the distance and the color-magnitude diagram (CMD) of the stream.

\subsubsection{Model}

To model the CMD of the stream, we use a probabilistic approach analogous to the
one described in Section~\ref{vlos}. In our data set, $\mathcal{D}$, each data
point ${\bf d_k}$ is now defined by its galactic longitude and by its PS1
$grizy_{P1}$ magnitudes ${\bf d_k}=\{l_k, g_k, r_k, i_k, z_k, y_k\}$.

Our data set contains only the Ophiuchus stream stars that were identified based
on spectroscopic data (i.e., velocity and metallicity). Thus, the set is
uncontaminated but very sparse and has a complicated spatial selection function.
Because of this, and because we are primarily interested in constraining the
distance of the stream, we focus on finding the isochrone(s) that match the
confirmed members in the color-magnitude space, and do not to model the
projected shape of the stream on the sky (for now, but see
Section~\ref{proper_motion_extent}).

To model the stream in color-magnitude space, we use a grid of theoretical
PARSEC isochrones\footnote{\url{http://stev.oapd.inaf.it/cmd}} (release v1.2S;
\citealt{bre12}; \citealt{che14}). Each isochrone $\mathcal{I}$ provides PS1
magnitudes
$m^\prime=g_{P1}^\prime,r_{P1}^\prime,i_{P1}^\prime,z_{P1}^\prime,y_{P1}^\prime$
for a single stellar population of age $t$, metal content $Z$, and parametrized
for the mass-loss on the red giant branch using the variable values of the
Reimers law parameter $\eta$ \citep{rei75, rei77}. At galactic longitude
$l=5\arcdeg$, the distance modulus of the stellar population is defined with
parameter $\overline{DM}$, and a gradient in distance modulus with galactic
longitude is modeled with parameter $\frac{dDM}{dl}$. The reddening is modeled
by adding the
$C_{ext}\left(\overline{E(B-V)} + \frac{d E(B-V)}{dl}\left(l-5\right)\right)$
term to isochrone magnitudes, where $\overline{E(B-V)}$ is the reddening at
galactic longitude $l=5\arcdeg$ and $\frac{d E(B-V)}{dl}$ is a possible
gradient in reddening. The extinction coefficients
$C_{ext}=3.172, 2.271, 1.682, 1.322,1.087$ for PS1 $grizy$ bands were taken from
Table 6 of \citet{sf11}.

To account for the fact that stellar evolution models are not perfect, we
introduce a parameter $\sigma_{iso}$ that models the uncertainty in PS1 $grizy$
magnitudes provided by PARSEC isochrones. In principle, one such parameter
should be used for each band, as the uncertainty in models may propagate
differently to different bands. We compared our results when a single
$\sigma_{iso}$ parameter is used for all bands, and when each band has its own
model uncertainty. We have found that our results do not change significantly
between these two cases, and have thus decided to use only one model magnitude
uncertainty parameter for all bands ($\sigma_{iso}$) when modeling the
color-magnitude diagram of the stream.

Given the above model of the stream, the likelihood of a data point $k$ is
\begin{equation}
    \mathcal{L}_k({\bf d_k} | \mathcal{I}_k) = \int_{\mathcal{I}_k} \prod_{m=g,r,i,z,y} \mathcal{N}(m_k | m^\prime, \sigma^\prime_{m_k})d\mathcal{I}_k, \label{likelihood}
\end{equation}
where
\begin{equation}
\mathcal{I}_k = \mathcal{I}(l_k | age, Z, \eta, \overline{E(B-V)}, \frac{dE(B-V)}{dl}, \overline{DM}, \frac{dDM}{dl}, \sigma_{iso})
\end{equation}
is the isochrone at the galactic longitude of data point $k$,
$\mathcal{N}(x|\mu,\sigma)$ is a normal distribution, and
$\sigma^\prime_{m_k}=\sqrt{\sigma_{m_k}^2 + \sigma_{iso}^2}$ is the sum of
uncertainties in the isochrone magnitude ($\sigma_{iso}$) and the observed
magnitude of data point $k$ ($\sigma_{m_k}$). The likelihood of all data points
can then be calculated by combining Equations~\ref{total_likelihood}
and~\ref{likelihood}. The integral in Equation~\ref{likelihood} implies
summation over an isochrone, that is, the observed $grizy_{P1}$ magnitudes are
compared to predicted $grizy_{P1}$ magnitudes over the entire isochrone.

\subsubsection{Priors\label{priors_section} on the CMD model}

Before we can calculate the probability of a model, we need to define the prior
probabilities of model parameters. Below, we list our priors and describe the
justification for each one. A summary of priors is given in Table~\ref{priors}.

Based on spectroscopic data, the Ophiuchus stream is metal-poor
(${\rm [Fe/H]=-1.95\pm0.05}$ dex) and $\alpha-$enhanced
(${\rm [\alpha/Fe]=0.4\pm0.1}$ dex). As shown by \citet{scs93}, the
$\alpha-$enhanced stellar population models are equivalent to scaled-solar ones 
with the same global metal content ${\rm [M/H]}$, where ${\rm [M/H]}$ for
$\alpha-$enhanced models can be calculated using their Equation 3
\begin{equation}
    [M/H]\approxeq [Fe/H] + \log_{10}(0.638\times10^{[\alpha/Fe]} + 0.362).
\end{equation}
For the chemical abundance of the Ophiuchus stream, the above equation implies
${\rm [M/H]}=-1.7\pm0.2$ dex. This means that we should adopt the normal
distribution $\mathcal{N}(\log_{10}(Z/Z_\sun)| -1.7, 0.2)$ as the prior
probability of metallicity $Z$ (where $Z_\sun=0.0152$ is the solar metal content
used by this perticular set of PARSEC isochrones). However, in the context of
cross-validating our analysis, we decided to replace the above metallicity prior
in favor of a (less informative) prior that is uniform in the
$0.0001 < Z < 0.0004$ range. Even though a less informative prior was adopted,
at the end of Section~\ref{CMD_posterior} we find a very impressive consistency
between the posterior distribution of metallicity $Z$ (obtained using CMD
fitting) and the spectroscopic estimate of $Z$ (see bottom panel of
Figure~\ref{DM_Z_posteriors}). In the end, it is important to note that our
results would not have changed significantly if we used the more informative
prior for metallicity content $Z$.

The presence of BHB stars, the ${\rm [Fe/H]}$ and the $\alpha-$enhancement of
the stream point to an old stellar population. Thus, for age we adopt a uniform
prior in the $8 < age/{\rm Gyr} < 13.5$ range.

Metal-poor and old populations have $\eta\sim0.4$ \citep{rpf88}. Therefore, for
the mass-loss parameter $\eta$ we adopt a uniform prior in the $0.2<\eta<0.5$
range. For the uncertainty in isochrone magnitudes, we adopt a prior that is
uniform in the $0 \leq \sigma_{iso} < 0.1$ mag range.

Based on the inspection of \citet{SFD98} and \citet{sch14} dust maps in the
Ophiuchus region, we adopt uniform priors for the reddening at $l=5\arcdeg$ and 
its gradient: $0.1 < \overline{E(B-V)} < 0.3$ mag and
$0 \leq \frac{dE(B-V)}{dl} < 0.1$ mag deg$^{-1}$. According to \citet{ber14b},
the Ophiuchus stream is located about $9\pm1$ kpc from the Sun. Thus, for
$\overline{DM}$ we adopted a uniform prior in the $14.2 < \overline{DM} < 15.2$ 
mag range (corresponding to the 7-11 kpc range). For the gradient in distance
modulus, a uniform prior in the
$\arrowvert\frac{dDM}{dl}\arrowvert < 0.5$ mag deg$^{-1}$ range is adopted.

\begin{deluxetable}{lll}
\tabletypesize{\scriptsize}
\setlength{\tabcolsep}{0.02in}
\tablecolumns{3}
\tablewidth{0pc}
\tablecaption{Prior probabilities of CMD parameters\label{priors}}
\tablehead{
    \colhead{Parameter} & \colhead{Prior type} & \colhead{Range}
}
\startdata
Age & uniform & 8 to 13.5 Gyr\\
Mass-loss parameter $\eta$ & uniform & 0.2 to 0.5 \\
Metallicity $Z$ & uniform & 0.0001 to 0.0004 \\
$\overline{E(B-V)}$ & uniform & 0.1 to 0.3 mag \\
$\frac{dE(B-V)}{dl}$ & uniform & 0 to 0.1 mag deg$^{-1}$ \\
$\overline{DM}$ & uniform & 14.2 to 15.2 mag\\
$\frac{dDM}{dl}$ & uniform & -0.5 to 0.5 mag deg$^{-1}$ \\
$\sigma_{iso}$ & uniform & 0 to 0.1 mag
\enddata
\end{deluxetable}

\subsubsection{Posterior distributions of CMD parameters\label{CMD_posterior}}

To efficiently explore the parameter space, we use the \citet{gw10} Affine
Invariant Markov chain Monte Carlo (MCMC) Ensemble sampler as implemented in the
{\em emcee} package\footnote{\url{http://dan.iel.fm/emcee/current/}} (v2.1,
\citealt{fm12}). We use 1000 walkers and obtain convergence\footnote{We checked 
for convergence of chains by examining the autocorrelation time of the chains
per dimension.} after a short burn-in phase of 100 steps per walker. The chains 
are then restarted around the best-fit value and evolved for another 4000 steps.
We find that the marginal posterior distributions of parameters are quite
Gaussian-like in shape, so we simply characterize their position using the
median, and their width using the central $68\%$ confidence limits (hereafter
CL; see Table~\ref{stream_parameters}).

\begin{deluxetable}{lr}
\tabletypesize{\scriptsize}
\setlength{\tabcolsep}{0.02in}
\tablecolumns{2}
\tablewidth{0pc}
\tablecaption{Ophiuchus stream parameters\label{stream_parameters}}
\tablehead{
\colhead{Parameter} & \colhead{Median and central $68\%$ CL$^a$}
}
\startdata
$\overline{v_{los}}$ & $289.1\pm0.5$ km s$^{-1}$ \\
$\frac{dv_{los}}{dl}$ & $4\pm1$ km s$^{-1}$ deg$^{-1}$ \\
${\rm [Fe/H]}$ & $-1.95\pm0.05$ dex \\
${\rm [\alpha/Fe]}$ & $0.4\pm0.1$ dex \\
\hline \\
Age & $12.7\pm0.4$ Gyr \\
Mass-loss parameter $\eta$ & $0.41\pm0.02$ \\
Metallicity $Z$ & $(2.3\pm0.2)\times 10^{-4}$ \\
$\overline{E(B-V)}$ & $0.19\pm0.01$ mag \\
$\frac{dE(B-V)}{dl}$ & $0.04\pm0.01$ mag deg$^{-1}$ \\
$\overline{DM}$ & $14.68\pm0.03$ mag \\
$\frac{dDM}{dl}$ & $-0.23\pm0.03$ mag deg$^{-1}$ \\
$\sigma_{iso}$ & $0.015\pm0.004$ mag \\
\hline \\
$l_{min}$ & $4.10_{-0.08}^{+0.01}$ deg \\
$l_{max}$ & $5.80_{-0.20}^{+0.05}$ deg \\
A & $31.369\pm0.008$ deg \\
B & $-0.73_{-0.04}^{+0.01}$ \\
C & $-0.08_{-0.05}^{+0.02}$ deg$^{-1}$ \\
Deprojected length & $1.6\pm0.3$ kpc \\
$\sigma_b$ & $6\pm1$ arcmin \\
$\overline{\mu_l}$ & $-6.0\pm0.4$ mas yr$^{-1}$ \\
$\frac{d\mu_l}{dl}$ & $-2.6\pm0.9$ mas yr$^{-1}$ deg$^{-1}$ \\
$\overline{\mu_b}$ & $2.3\pm0.4$ mas yr$^{-1}$ \\
$\frac{d\mu_b}{dl}$ & $2.1\pm0.8$ mas yr$^{-1}$ deg$^{-1}$ \\
$\sigma_{pm}$ & $0.8\pm0.5$ mas yr$^{-1}$ \\
$N_{stars}$ ($i_{P1} < 20$) & $193\pm25$ stars \\
\hline \\
Pericenter & $3.50\pm0.03^b$ kpc \\
Apocenter & $17.5\pm0.3$ kpc \\
Eccentricity & $0.67\pm0.01$ \\
Orbital period & $360\pm5$ Myr \\
Radial period & $245\pm3$ Myr \\
Vertical period & $356\pm5$ Myr \\
Mass of the progenitor & $\sim2\times10^{4}$ $M_\sun$ ($>4\times10^{3}$ $M_\sun$) \\
\enddata
\tablenotetext{a}{The median and the central 68\% confidence limits are measured
from marginal posterior distributions.}
\tablenotetext{b}{The uncertainties in orbital parameters do not account for the
uncertainties in the assumed potential.}
\end{deluxetable}

We find the stream to be $12.7\pm0.4$ Gyr old and to have a distance modulus of
$14.68\pm0.03$ mag (i.e., a distance of 8.6 kpc) at galactic longitude
$l=5\arcdeg$ (top panel of Figure~\ref{DM_Z_posteriors}). Most importantly, we
detect a gradient of $-0.23\pm0.03$ mag deg$^{-1}$ in distance modulus. This
gradient is inconsistent with zero (i.e., with the no gradient hypothesis) at a 
$7\sigma$ level, and agrees in sign with the \citet{ber14b} conclusion that the 
eastern part of the stream is closer to the Sun
(Figure~\ref{gl_vs_distance}). A comparison of CMDs of the Ophiuchus
stream and field stars is shown in Figure~\ref{CMD_models}. The CMD of field
stars (grayscale pixels) was obtained by binning the $g_{P1}-i_{P1}$ colors and
$i_{P1}$-band magnitudes of stars located more than $18\arcmin$ from the
equator\footnote{See Section 3 of \citealt{ber14b} for its definition.} of the
Ophiuchus stream.

\begin{figure}
\plotone{DM_Z_posteriors.pdf}
\caption{
Marginal posterior distributions of distance modulus at $l=5\arcdeg$ ({\em top})
and metallicity $Z$ ({\em bottom}). In the bottom panel, the solid vertical line
shows the mean metallicity $Z$ measured from spectroscopy
(${\rm [Fe/H]=-1.95\pm0.05}$ dex, ${\rm [\alpha/Fe]} = 0.4\pm0.1$ dex), while
the dashed lines show the uncertainty in the spectroscopic estimate of $Z$.
\label{DM_Z_posteriors}}
\end{figure}

\begin{figure}
\plotone{gl_vs_distance.pdf}
\caption{
Heliocentric distance of the Ophiuchus stream as a function of galactic
longitude $l$. The thick solid line shows the most probable model
($DM(l)=-0.23(l-5)+14.68$ mag). To illustrate the uncertainty in the most
probable model, the thin semi-transparent red lines show 200 models sampled from
the posterior distribution. The vertical dashed lines show the likely extent of
the stream (see Section~\ref{proper_motion_extent}). The white circles plotted
on top of the solid line show the positions of 14 confirmed stream members,
where their distance modulus was calculated using the most probable model of
$DM(l)$.
\label{gl_vs_distance}}
\end{figure}

\begin{figure}
\plotone{CMD_models.pdf}
\caption{
The $g_{P1}-i_{P1}$ vs.~$i_{P1}$ color-magnitude diagram showing the most
probable isochrone (yellow thick line) and 200 isochrones randomly sampled from 
the stream's full posterior distribution (semi-transparent dark red thin lines).
The isochrones have been shifted to match the distance of the stream at
$l=5\arcdeg$. The grayscale pixels show the density distribution of field stars 
in this diagram (i.e., their probability density function). For illustration
only, the color and magnitude of observed stars have been corrected for
gradients in distance modulus and reddening, using median $\frac{dDM}{dl}$ and
$\frac{dE(B-V)}{dl}$ values.
\label{CMD_models}}
\end{figure}

The gradient in distance modulus could be due to a gradient in reddening.
However, while the two gradients are strongly anti-correlated (see
Figure~\ref{DeltaDM_vs_DeltaEBV}), they are still well-constrained by our data,
as demonstrated by their $95\%$ CLs that are narrower than their prior
probability distributions (of $\arrowvert\frac{dDM}{dl}\arrowvert< 0.5$ mag
deg$^{-1}$ and $0 \leq \frac{dE(B-V)}{dl} < 0.1$ mag deg$^{-1}$, respectively).

\begin{figure}
\plotone{DeltaDM_vs_DeltaEBV.pdf}
\caption{
The joint posterior distribution of gradients in distance modulus and reddening
(grayscale pixels). The $68\%$ ($1\sigma$) and $95\%$ ($2\sigma$) CL  contours
are shown as solid and dashed lines, respectively.
\label{DeltaDM_vs_DeltaEBV}}
\end{figure}

In Section~\ref{priors_section}, we adopted a uniform prior for the metallicity
content $Z$ in order to test the predictive power our dataset. As shown in the
bottom panel of Figure~\ref{DM_Z_posteriors}, the peak of the marginal posterior
distribution of $Z$ is consistent with the mean value of $Z$ estimated from
spectroscopic data (solid vertical line), and the distribution is even narrower 
than the distribution of $Z$ estimated from spectroscopy (dashed vertical
lines). This result demonstrates the predictive power of our data. It shows how 
a combination of good coverage of the CMD, PS1 photometry, and detailed modeling
can provide an accurate and precise estimate of the metallicity of single
stellar populations.

\subsection{Modeling the proper motion and the extent of the stream\label{proper_motion_extent}}

The longitude-dependent CMD model we have built in Section~\ref{CMD}, and the
luminosity functions associated with the model, allow us to assign a likelihood
that a star is a member of the Ophiuchus stream, based on the star's galactic
longitude $l$, $g-i$ color, and $i$-band magnitude. The distribution of field
stars in the $g-i$ vs.~$i$ color-magnitude diagram (grayscale pixels in
Figure~\ref{CMD_models}), on the other hand, enables us to estimate the
likelihood that a star is associated with the field. As we show in this section,
these two probability density functions, when combined with positional and
proper motion data, can be used to simultaneously trace the extent of the
Ophiuchus stream and determine its proper motions across the sky.

In principle, we could measure the proper motion of the Ophiuchus stream using
the proper motion of its confirmed members. However, since our sample of
confirmed members contains only 14 stars, there is a possibility that one or two
stars with incorrectly measured proper motions may bias the results. As an
example, stream member ``rgb4'' is clearly an outlier in proper motion as it has
$\mu_l\sim-30$ mas yr$^{-1}$, while the remaining members have $\mu_l\sim-6$ mas
yr$^{-1}$. A visual inspection of digitized photographic plates has revelead
that ``rgb4'' is blended with a neighbor of similar brightness, which affects
the measured position of the star and its proper motion.

Fortunately, we do not need to rely only on confirmed members and can instead
use a much larger sample of stars in the vicinity of the Ophiuchus stream to
constrain its proper motion and extent. As we detail below, we use a
probabilistic approach (see Sections~\ref{vlos} and~\ref{CMD}) and model the
distribution of stars simultaneously in coordinate, proper motion, and
color-magnitude space as a mixture of stream and field (i.e., non-stream) stars.
Even though we do not {\em a priori} know which star is a true member of the
stream, we assume that as an ensemble, the stream stars have certain
characteristics which make them distinguishable from field stars (e.g., common
proper motion, distance, position on the sky and in the CMD), and that the
{\em scatter} in these characteristics is sufficiently small to overcome the
fact that there are a lot more field than stream stars. The narrow width of
the stream in color-magnitude (Figure~\ref{CMD_models}) and coordinate space
(Figure~1 of \citealt{ber14b}) support this assumption. After all, if the stream
did not have these characteristics, it likely would not have been detected by
\citet{ber14b} in the first place.

Assuming the stream extends between galactic longitudes $l_{min}$ and $l_{max}$,
the likelihood that a star with galactic longitude $l_k$, latitude $b_k$, proper
motions in galactic coordinates of $\mu_{l,k}$ and $\mu_{b,k}$, color $(g-i)_k$
and magnitude $i_k$ is drawn from the mixture model, is equal to
\begin{equation}
\mathcal{L}({\bf d_k} | \theta) = f(l_k)\mathcal{L}({\bf d_k} | \theta_{stream}) + (1-f(l_k))\mathcal{L}({\bf d_k} | \theta_{field}),\label{mix_model}
\end{equation}
where ${\bf d_k} \equiv \{l_k, b_k, \mu_{l,k}, \mu_{b,k}, (g-i)_k, i_k \}$
contains measurements for data point (star) $k$, and
$\theta\equiv\{\theta_{stream}, \theta_{field}\}$ contains parameters that model
the distribution of stream and field stars, respectively. The parameter $f$
specifies the fraction of stars in the stream (out of all stars between
$l_{min}$ and $l_{max}$) and is $f(l_k) \in [0,1]$ for $l_{min} < l_k< l_{max}$,
otherwise, it is $f(l_k)=0$.

The likelihood $\mathcal{L}({\bf d_k} | \theta_{stream})$ is a product of
spatial likelihood $\mathcal{L}(l_k, b_k | \theta^{spatial}_{stream})$,
proper motion likelihood
$\mathcal{L}(l_k, \mu_{l,k},\mu_{b,k}|\theta^{pm}_{stream})$, and the
color-magnitude likelihood $\mathcal{L}(l_k,(g-i)_k,i_k| \theta^{CM}_{stream})$
\begin{equation}
\begin{split}
\mathcal{L}({\bf d_k} | \theta_{stream})=\mathcal{L}(l_k, b_k | \theta^{spatial}_{stream}) \\
\times\mathcal{L}(l_k, \mu_{l,k},\mu_{b,k}|\theta^{pm}_{stream}) \\
\times\mathcal{L}(l_k, (g-i)_k, i_k| \theta^{CM}_{stream}).
\end{split}
\end{equation}
The likelihood for field stars, $\mathcal{L}({\bf d_k} | \theta_{field})$, has
the same decomposition.

In galactic coordinates, the distribution of stream stars is modeled as a
parabola with a Gaussian width $\sigma_b$ in the latitude direction, that is
\begin{equation}
    \mathcal{L}(l_k, b_k | \theta^{spatial}_{stream}) = \mathcal{N}(b_k | \nu(l_k), \sigma_b),
\end{equation}
where $\nu(l_k | A, B, C)=A + B(l_k-5) + C(l_k-5)^2$ is the galactic latitude of
the equator of the stream. The spatial distribution of field stars is modeled as
a straight line with a Gaussian width $\sigma^\prime_b$
\begin{equation}
    \mathcal{L}(l_k, b_k | \theta^{spatial}_{field}) = \mathcal{N}(b_k | \nu^\prime(l_k), \sigma^\prime_b),
\end{equation}
where $\nu^\prime(l_k | A^\prime, B^\prime)=A^\prime + B^\prime(l_k-5)$.

At galactic longitude $l=5\arcdeg$, the stream is assumed to have proper motion
$\overline{\mu_l}$ and $\overline{\mu_b}$, with possible gradients in proper
motion of $\frac{d\mu_l}{dl}$ and $\frac{d\mu_b}{dl}$ (i.e., gradients as a
function of galactic longitude). The proper motion likelihood of stream stars is
then
\begin{equation}
\begin{split}
\mathcal{L}(l_k,\mu_{l,k},\mu_{b,k}|\theta^{pm}_{stream})= \\
\mathcal{N}(\mu_{l,k}| \mu_l(l_k), \sigma^\prime_k) \\
\times\mathcal{N}(\mu_{b,k}| \mu_b(l_k), \sigma^\prime_k),
\end{split}
\end{equation}
where $\mu_l(l_k)=\frac{d\mu_l}{dl}\left(l_k-5\right) + \overline{\mu_l}$ and
$\mu_b(l_k)=\frac{d\mu_b}{dl}\left(l_k-5\right) + \overline{\mu_b}$ are the
predicted proper motions of the stream at galactic longitude $l_k$, and
$\sigma^\prime_k = \sqrt{\sigma_{pm}^2 + \sigma^2_{\mu,k}}$ is the quadratic sum
of the intrinsic proper motion dispersion and the uncertainty in the
corresponding proper motion of data point $k$. The purpose of parameter
$\sigma_{pm}$ is to account for any additional scatter in proper motions (e.g., 
due to unaccounted errors). The proper motion likelihood of field stars has the
same form (but different parameters) as the proper motion likelihood of stream
stars.

The likelihood that a star is drawn from the stream's CMD is defined as
\begin{equation}
\begin{split}
\mathcal{L}(l_k, (g-i)_k, i_k| \theta^{CM}_{stream})=\zeta \int\int \mathcal{N}((g-i)^\prime_k| g-i, \sigma_{(g-i)_k}) \\
\times \mathcal{N}(i^\prime_k | i, \sigma_{i_k})\mathcal{I}(g-i, i)d(g-i)di,\label{CM_likelihood}
\end{split}
\end{equation}
where
$\sigma_{(g-i)_k}$ and $\sigma_{i_k}$ are the uncertainty in color and magnitude
of data point $k$, and $\zeta$ is a normalization constant calculated such that
the integral of Equation~\ref{CM_likelihood} over the considered region of CM
space is unity. To account for the gradient in reddening,
$(g-i)^\prime_k=(g-i)_k - 1.49\frac{dE(B-V)}{dl}(l_k-5)$ and
$i^\prime_k=i_k - 1.682\frac{dE(B-V)}{dl}(l_k-5) - \frac{dDM}{dl}(l_k-5)$ are
the $g_{P1}-i_{P1}$ color and $i_{P1}$-band magnitude of data point $k$ offset
by the difference in reddening between galactic longitudes $l_k$ and
$l=5\arcdeg$, where $\frac{dE(B-V)}{dl}=0.04$ mag deg$^{-1}$ is the most
probable gradient in reddening (see Table~\ref{stream_parameters}). Note that
magnitude $i_k$ is also offset by the difference in distance modulus between
galactic longitudes $l_k$ and $l=5\arcdeg$ (to account for the gradient in DM), 
where $\frac{dDM}{dl}=-0.23$ mag deg$^{-1}$.

In Equation~\ref{CM_likelihood}, $\mathcal{I}(g-i, i)$ is the probability
density function (PDF) of the Ophiuchus stream in the $g_{P1}-i_{P1}$
vs.~$i_{P1}$ color-magnitude space at galactic longitude $l=5\arcdeg$. This PDF
was constructed by sampling isochrones from the stream's CMD model
(Section~\ref{CMD_posterior}), multiplying them with their luminosity functions,
and then summing them up in a binned $g_{P1}-i_{P1}$ vs.~$i_{P1}$
color-magnitude diagram. The likelihood that a star is drawn from the field CMD
has the same form as Equation~\ref{CM_likelihood}, except the PDF for the field
stars is different (see grayscale pixels in Figure~\ref{CMD_models}).

In total, our model contains 20 parameters. For all of the parameters, we have
adopted uniform priors within certain bounds. The allowed ranges of model
parameters were determined by examining positions and proper motions of
confirmed members and other stars. In addition to adopted priors, we also
require that the parameters satisfy following constraints:
\begin{itemize}
\item the spatial width of the stream must be smaller or equal than the width of
the spatial distribution of field stars: $\sigma_b \leq \sigma^\prime_b$
\item the additional scatter in proper motion of stream stars must be smaller
than the scatter in proper motions of field stars:
$\sigma_{pm} \leq \sigma^\prime_{pm}$, and
\item the galactic latitudes of confirmed members ($b^{conf}_k$) must be within
$3\sigma_b$ of the equator of the stream:
$|\nu(l^{conf}_k) - b^{conf}_k| \leq 3\sigma_b$,
where $\nu(l^{conf}_k)$ is the galactic latitude of the stream at the position
of confirmed members, and $1 \leq k \leq 14$.
\end{itemize}

As our data set, we use stars brighter than $i_{P1}=20$ mag with measured proper
motions, and located in a $2\times2$ deg$^2$ area centered on the Ophiuchus
stream. To explore the parameter space, we use 200 {\em emcee} walkers and
obtain convergence after a short burn-in phase of 100 steps. The chains are then
restarted around the best-fit value and evolved for another 2000 steps. The
median values and central $68\%$ confidence limits of stream parameters are
listed in Table~\ref{stream_parameters}.

\begin{figure}
\plotone{gl_vs_gb_track.pdf}
\caption{
The extent of the Ophiuchus stream in galactic coordinates. The thick solid line
shows the most probable model ($b(l)=31.369-0.73(l-5)-0.08(l-5)^2$ deg) for the
equator of the stream. To illustrate the uncertainty in the most probable
model, the thin semi-transparent red lines show 200 models sampled from the
posterior distribution. The vertical dashed lines show the likely extent of the
stream (see Section~\ref{proper_motion_extent}). The yellow points show the
positions of confirmed members and the arrow indicates the direction of movement
of the stream. The pixels show the probability-weighted number density map of
region, smoothed using a $6\arcmin$-wide Gaussian filter.
\label{gl_vs_gb_track}}
\end{figure}

We find the stream to be confined between galactic longitudes of 4.1 deg and 5.8
deg (Figure~\ref{gl_vs_gb_track}). When combined with the distance of the
stream (Figure~\ref{gl_vs_distance}), this result implies that the
deprojected length of the stream is $1.6\pm0.3$ kpc. Thus, the stream is very
foreshortened in projection, by a ratio of $6:1$. The galactic latitude of the
equator of the stream is at $b_{stream}(l) = 31.369 - 0.73(l-5) - 0.08(l-5)^2$
deg, and the width of the stream is $\sigma_b=6\pm1$ arcmin (in the galactic
latitude direction). In direction perpendicular to the stream's equator, the
stream is $\sim3.5\arcmin$ wide, which is similar to width of $\sim3\arcmin$
measured by \citet{ber14b}. Since the stream is very foreshortened, the
conversion and deprojection of the angular width is not a trivial matter, and
for this reason, we choose not provide an estimate of the physical width of the 
stream.

% XXX: Provide some estimate of the physical width.

Based on a star's galactic longitude, color, magnitude, and proper motion, and
given the model of the stream derived so far, we can evaluate its probability
that it is a member of the Ophiuchus stream. We have done so for all of the
stars in the vicinity of the Ophiuchus stream and have created a
probability-weighted number density map, shown as grayscale pixels in
Figure~\ref{gl_vs_gb_track}. An inspection of the number density map did
not reveal a significant overdensity of stars along the stream that would
indicate the presence of a progenitor.

\begin{figure}
\plotone{gl_vs_pm.pdf}
\caption{
Proper motion of the Ophiuchus stream as a function of galactic longitude $l$,
in the longitude ({\em top}) and latitude directions ({\em bottom}). The thick
solid lines show the most probable models ($\mu_l(l)=-2.6(l-5)-6.0$ mas
yr$^{-1}$, $\mu_b(l)=2.3(l-5)+2.3$ mas yr$^{-1}$). To illustrate the uncertainty
in most probable models, the thin semi-transparent red lines show 200 models
sampled from respective posterior distributions. For comparison, the
semi-transparent gray lines show the proper motion of field stars. The vertical 
dashed lines show the likely extent of the stream (see
Section~\ref{proper_motion_extent}).
\label{gl_vs_pm}}
\end{figure}

Our data indicate that the proper motion of the stream changes as a function of
galactic longitude (Figure~\ref{gl_vs_pm}). The gradients in proper motion
are significant at $\gtrsim2\sigma$ level, and while their absolute values are
similar ($\sim2$ mas yr$^{-1}$ deg$^{-1}$), the gradients have opposite signs.
For comparison, the gradients in proper motions of field stars are consistent
with zero at $1\sigma$ level ($\frac{d\mu^\prime_l}{dl}=0.12_{-0.11}^{+0.05}$
mas yr$^{-1}$ deg$^{-1}$ and $\frac{d\mu^\prime_b}{dl}=0.14_{-0.13}^{+0.04}$ mas
yr$^{-1}$ deg$^{-1}$). Overall, the proper motions of field stars are consistent
with apparent motions of a population at $\sim9$ kpc, due to the motion of the
Sun around the Galaxy.

The stream parameters we have obtained so far can be used to place a lower limit
on the mass of the initial population of the Ophiuchus stream. The fraction of
stars $f$ in the stream between longitudes $l_{min}$ and $l_{max}$, can be
converted to the number of stars in the stream, $N_{stars}$. We find that there
are $N_{stars}=193\pm25$ stars brighter than $i_{P1}=20$ mag in the Ophiuchus
stream. If we adopt the luminosity function associated with the most probable
CMD model of the stream and assume \citet{kro98} initial mass function (not
corrected for binarity), this number of stars implies that the initial
population of the Ophiuchus stream had to have a mass of at least
$M_{init}=(4.0\pm0.5)\times10^3$ $M_{\sun}$.

\section{Orbit of the Ophiuchus stream}\label{orbit}

The data and models of the stream obtained in previous sections now enable us to
constrain the orbit of the Ophiuchus stream. For this purpose we use
{\em galpy}\footnote{\url{http://github.com/jobovy/galpy}}, a package for
galactic dynamics written in {\em Python} programming language \citep{bov15}.

To take full advantage of the stream constraints derived so far, we randomly
draw 200 line-of-sight velocity, CMD, position, and proper motion models from
posterior distributions obtained in previous sections, and perform orbit fitting
on data sets created from these models. Each data set consists of 100 data
points uniformly sampled in galactic longitude from $l_{min}$ to $l_{max}$. Each
data point is defined by its position and proper motion in galactic coordinates,
heliocentric distance, and line-of-sight velocity. To emulate the width of the
stream, we assign an uncertainty of $\sigma_b$ to positions of data points. To
all data points we assign a 3\% uncertainty in distance, 2 km s$^{-1}$
uncertainty in velocity, and 2 mas yr$^{-1}$ of uncertainty in proper motions.
We have verified that our results do not change significantly if these
uncertainties are changed. To convert the observed values into 3D positions and
velocities, {\em galpy} assumes the circular velocity at the solar radius is 220
km s$^{-1}$, the Sun is located 8 kpc from the Galactic center, and Sun's motion
in the Galaxy is (-11.1, 244, 7.25) km s$^{-1}$ \citep{sbd10, bov12}.

The orbits are integrated in the default {\em galpy} potential, called
MWPotential2014 (Table 1 of \citealt{bov15}). This potential
consists of a bulge modeled as a power-law density profile that is exponentially
cutoff with a power-law exponent of -1.8 and a cut-off radius of 1.9 kpc, a
Miyamoto-Nagai disk, and a dark-matter NFW halo. MWPotential2014 is consistent
with a large variety of dynamical constraints on the potential of the Milky Way,
ranging from the bulge to the outer halo.

\begin{figure*}
\plottwo{gl_vs_rv_orbit.pdf}{gl_vs_distance_orbit.pdf}
\plottwo{gl_vs_gb_track_orbit.pdf}{gl_vs_pm_orbit.pdf}
\caption{
This plot compares line-of-sight velocities ({\em top left}), distances
({\em top right}), positions ({\em bottom left}), and proper motions
({\em bottom right}) calculated by {\em galpy} (thin blue lines) with models
derived from observations (thin red lines). The observed and calculated values
are consistent within uncertainties, with proper motions being the only
exception.
\label{galpy_orbit_comparison}}
\end{figure*}

The line-of-sight velocities, heliocentric distances, positions, and proper
motions predicted by {\em galpy} orbits are shown as thin semi-transparent blue
lines in Figure~\ref{galpy_orbit_comparison}. Overall, the observed and
predicted mean values and gradients agree within uncertainties. This agreement
is not trivial. While there is always an orbit that will fit a {\em single} star
in some potential, the same is not true for a {\em stream} of stars. For
example, given the observed gradients and mean values in proper motion,
distance, and position, the observed gradient in line-of-sight velocity has to
be positive, otherwise, there is a strong discrepancy with the velocity
predicted by the most probable orbit. Similarly, the observed gradient in
distance modulus has to have a negative sign, otherwise a good orbit fit cannot 
be achieved. The most noticeable disagreement is between observed and predicted
proper motions (bottom right panel of Figure~\ref{galpy_orbit_comparison}), with
the observed proper motions having a steeper gradient (by about a factor of
two). While this discrepancy is significant, the mean proper motion of the
stream (towards the Galactic center and away from the plane) is consistent with
the extent and orientation of the stream (i.e., curving towards the Galactic
center and moving away from the plane). Thus, this discrepancy does not
significantly affect the results presented below.

\begin{figure}
\plotone{orbit_XYZ.pdf}
\caption{
The orbit of the Ophiuchus stream in the past 370 Myr (about one orbital
period), shown in the galactocentric Cartesian coordinate system. Note the
pericenter passage at $t\approx-250$ Myr (solid square). Near this point in
time, the stream was also passing through the disk ($Z\sim0$ kpc, see the top
left panel) and was experiencing strong tidal forces due to disk shocking
(see Figure~\ref{tidal_force}).
\label{orbit_XYZ}}
\end{figure}

Figure~\ref{orbit_XYZ} illustrates the orbit of the Ophiuchus stream in the past
370 Myr. We find that the stream has a relatively short orbital period of
$360\pm5$ Myr, and a fairly eccentric orbit ($e=0.67\pm0.01$), with a
pericenter of $3.50\pm0.03$ kpc and an apocenter of $17.5\pm0.3$ kpc. About 10
Myr ago, the stream passed through its pericenter and now it is moving away from
the Galactic plane and towards the center. Regarding uncertainties on orbital
parameters, we note that they {\em do not} account for uncertainties in the
assumed MWPotential2014 potential.

\section{Time of disruption}

In the Introduction, we said that the short length of the Ophiuchus stream
suggests that its progenitor must have been disrupted fairly recently. As we
have shown in Section~\ref{proper_motion_extent}, part of the reason the stream
is so short {\em in projection}, is the viewing angle -- we are observing the
stream almost end-on.

Even when the projection effects are taken into account, the deprojected length
is still fairly short, only 1.6 kpc. For comparison, the second shortest stellar
stream is the Pisces stream \citep[also known as the Triangulum stream,
\citealt{bon12}]{pisces} with a length of $\sim5.5$ kpc. Therefore, the length
of the stream still suggests that the stream formed recently, that is, it
suggests that the progenitor was recently disrupted.

As the progenitor of the Ophiuchus stream orbited the Galaxy, it would have
experienced the tidal force of the Galactic potential. This force can strip
stars from the progenitor and it could have been strong enough to completely
disrupt the progenitor. 

In order to examine the influence of the tidal force, we have calculated its
magnitude as a function of time for the most probable orbit of the Ophiuchus
stream. The magnitude of the tidal force was calculated by finding the largest
eigenvalue of the following matrix
\begin{eqnarray}
J & = &
\left(
\begin{array}{cc}
    \frac{d^2\Phi}{dR^2} & \frac{d^2\Phi}{dRdZ} \\
    \frac{d^2\Phi}{dZdR} & \frac{d^2\Phi}{dZ^2}
\end{array}
\right),
\end{eqnarray}
where $\Phi$ is value of the {\em galpy} MWPotential2014 potential at the
position of the progenitor, and $R$ and $Z$ are coordinates in the cylindrical
galactocentric system. The result is shown in Figure~\ref{tidal_force}.

\begin{figure}
\plotone{tidal_force_vs_time_max_eigen.pdf}
\caption{
The tidal force acting on the Ophiuchus stream, normalized to the maximum tidal
force in the past 2 Gyr. The narrow peaks correspond to passages through the
disk (i.e., disk shocking, \citealt{osc72}) and the broader peaks correspond to
passages through the pericenter.
\label{tidal_force}}
\end{figure}

We find that the tidal force is the strongest during pericenter+disk passages,
and the progenitor of the stream could have been disrupted during one of those
passages. To find if the progenitor was disrupted during one of these passages,
we can use {\em galpy}.

\begin{figure}
\plotone{prob_map.pdf}

\plotone{galpy_stream_map.pdf}

\plotone{nemo_stream_map.pdf}
\caption{
A comparison of the observed number density map of the stream ({\em top}), a map
created from a mock stream generated using {\em galpy} ({\em middle}), and a map
created from a N-body stream generated using NEMO. In all panels, the solid line
shows the most probable position of the stream and the dashed lines illustrate
its $1\sigma$ width. The mock {\em galpy} stream was generated assuming time of
disruption $t_{dis}=170$ Myr, and velocity dispersion $\sigma_{v}=0.4$ km
s$^{-1}$. Note a good agreement between the length and width of the observed,
{\em galpy} and NEMO streams.
\label{observed_vs_mock_comparison}}
\end{figure}

Given an orbit, the time of disruption $t_{dis}$, and the velocity dispersion of
the stream $\sigma_v$, {\em galpy} can generate a mock stream using the modeling
framework of \citet{bov14}. Observationally, for a fixed $\sigma_v$, $t_{dis}$
is proportional to the stream's length (i.e., older streams are longer). In
Section~\ref{vlos}, we measured $s=0.4$ km s$^{-1}$ as the velocity dispersion
of the stream. We fix $\sigma_v$ to that value, and find that $t_{dis}\sim170$
Myr provides a good match between the observed and mock streams (see
Figure~\ref{observed_vs_mock_comparison}). A much longer or a much shorter time
of disruption produces a much longer or shorter stream that is not consistent
with observations. This result, and the fact that there was a disk+pericenter
passage 250 Myr ago, strongly suggest that the stream formed about 250 Myr ago
(i.e., that the progenitor was disrupted at that time).

As shown by \citet{sb13}, the velocity dispersion of the stream scales with the 
mass of the progenitor as $\sigma_v \propto M_{dyn}^{1/3}$. In {\em galpy}, the
model used for stream generation was calibrated using a progenitor of mass
$M_{dyn}=2\times10^4$ $M_\sun$ which, after it was disrupted, formed a stream
with a velocity dispersion of $\sigma_v=0.365$ km s$^{-1}$ \citep{bov14}. Using
the above scaling relation, we find that the progenitor of the Ophiuchus stream
had a mass of $M_{dyn}\sim2\times10^4$ $M_\sun$.

It is important to note that {\em galpy} creates only the stream, and that stars
associated with the progenitor are not part of the mock stream (i.e., are not
shown in the middle panel of Figure~\ref{observed_vs_mock_comparison}). This is
the reason an overdensity of stars (i.e., the progenitor) is not visible in the
mock {\em galpy} stream.

To create a more realistic stream that includes a progenitor, we use the
gyrfalcON code \citep{deh00,deh02} in the NEMO toolkit \citep{teu95}. We set up
the progenitor as a King cluster \citep{kin66} with a mass of $1\times10^4$
$M_\sun$, tidal radius of 94 pc and a ratio of the central potential to the
velocity dispersion squared of 2.0. The cluster is sampled using 20,000
particles and is evolved for $\sim375$ Myr in the MWPotential2014 potential. The
initial conditions are those at the $t\sim-375$ Myr apocenter of the most
probable orbit of the Ophiuchus stream.

The number density map of the N-body stream created using gyrfalcON is shown in
the bottom panel of Figure~\ref{observed_vs_mock_comparison}. The width and the
length of the N-body stream match the observed stream fairly well. For
comparison, if the cluster is evolved starting from the apocenter at
$t\sim-870$ Myr, the resulting stream is much longer and inconsistent with
observations. Based on this more realistic simulation, we conclude that the
Ophiuchus stream likely formed about 250 Myr ago and that its progenitor was
fully disrupted during a single disk+pericenter passage. If this scenario is
correct, then the answer to the question ``Where is the progenitor of the
Ophiuchus stream?'' is simple -- the Ophiuchus stream is all that is left of
the progenitor.

To reproduce the lack of a visible progenitor, we had to use a fairly large
tidal radius of 94 pc. This tidal radius is large, but is not unknown for a
globular cluster. For example, globular cluster Pal 3 has a mass of
$3\times10^4$ $M_\sun$ and a tidal radius of 108 pc (Table 1 of
\citealt{mpv14}). However, we must point out that Pal 3 is located 90 kpc from
the Galactic center, unlike the Ophiuchus stream that has an apocenter at
$\sim18$ kpc.

If the Ophiuchus stream is all that is left of the progenitor, and if the
progenitor was disrupted 250 Myr ago as our simulations suggest, how did the
progenitor remain bound for 13 Gyr (which is the age of the stream's
population) on this fairly short-period orbit? If the progenitor was on this
orbit for the past 13 Gyr, it would have passed through the disk more than 40
times. What was so special about the disk passage 250 Myr ago, compared to all
the previous disk passages? On the other hand, if the progenitor was not always 
on this orbit, and if its apocentar was more distant (like that of Pal 3), how
did its orbit change? Even though we believe we have resolved a few mysteries
related to the Ophiuchus stream, more mysteries remain.

\section{Conclusions and Summary}\label{conclusions}

In this paper, we have presented follow-up spectroscopy and an astrometric and
photometric analysis of the Ophiuchus stellar stream in the Milky Way, recently
discovered by \citet{ber14b} in PS1 data. We have been able to put together a
comprehensive, empirical description of the Ophiuchus stream in phase space: we
succeeded in determining the mean phase-space coordinates in all six dimensions,
along with the gradients of those coordinates along the stream (see
Table~\ref{stream_parameters} for a summary of stream parameters).

Overall, phase-space data along the stream can be well matched by an orbit in a
fiducial Milky Way potential: Ophiuchus appears $6:1$ foreshortened in
projection; it is on a highly inclined orbit with only a 360 Myr orbital period;
it is receding from us at nearly 300 km s$^{-1}$ and has just passed its
pericenter at $\sim3$ kpc from the Galactic center. This makes Ophiuchus the
innermost stellar stream known in our Galaxy. It is also the only known
kinematically-cold stellar stream to be seen nearly end-on.

The homogeneously metal-poor (${\rm [Fe/H]}=-2.0$ dex), $\alpha$-enhanced
(${\rm [\alpha/Fe]\sim0.4}$ dex) and old stellar population ($\sim13$ Gyr old),
and the small line-of-sight velocity dispersion we found ($<1$ km s$^{-1}$),
confirm the notion that the progenitor of the stream was a globular cluster,
likely with a mass of $\sim2\times10^4$ $M_\sun$ (or at least greater than
$\sim4\times10^3$ $M_\sun$). In this respect, the Ophiuchus and the GD-1 stream
\citep{gd06, kop10} can be considered identical twins, as they have the same
metallicity, the same mass, and no visible progenitors.

Our analysis, however, leaves a number of questions open. First, the most
probable orbit in the fiducial potential is not quite able to match the proper
motions and their gradients along the stream. A thorough exploration whether
there are axiymmetric or non-axisymmetric potentials that might be able to
remedy this tension remains to be done. The present analysis does not yet use
the stream phase-space data to provide new constraints on the Galactic
potential. In principle, Ophiuchus stream can provide constraints on the
Galactic potential at about 5 kpc above the Galactic center, a location where
few other constraints exist. Second, even though our analysis indicates that the
stream only disrupted about 250 Myr ago, no indication of a progenitor remnant
have been identified. Orbit integrations show that disk shocking during a
pericenter passage delivers the most damaging tidal forces. But we cannot offer 
a clear explanation on how Ophiuchus, with its 13-Gyr-old population of stars,
survived about 40 such passages on its present orbit, only to be disrupted 250
Myr ago. There are two obvious avenues for an explanation: either Ophiuchus did 
not spend most of its life on its present orbit, or a particular combination of 
pericenter and disk passage, or even passage through the Galactic bar, proved
fatal to the clusters structural integrity just in the recent past. We expect
that detailed N-body simulations will provide a more definitive answer to these 
questions and plan to pursue this approach in the near future.

\acknowledgments

B.S.~acknowledges funding from the European Research Council under the European
Union’s Seventh Framework Programme (FP 7) ERC Grant Agreement
n.~${\rm [321035]}$. C.I.J.~gratefully acknowledges support from the Clay
Fellowship, administered by the Smithsonian Astrophysical Observatory. The
Pan-STARRS1 Surveys (PS1) have been made possible through contributions of the
Institute for Astronomy, the University of Hawaii, the Pan-STARRS Project
Office, the Max-Planck Society and its participating institutes, the Max Planck
Institute for Astronomy, Heidelberg and the Max Planck Institute for
Extraterrestrial Physics, Garching, The Johns Hopkins University, Durham
University, the University of Edinburgh, Queen's University Belfast, the
Harvard-Smithsonian Center for Astrophysics, the Las Cumbres Observatory Global
Telescope Network Incorporated, the National Central University of Taiwan, the
Space Telescope Science Institute, the National Aeronautics and Space
Administration under Grant No.~NNX08AR22G issued through the Planetary Science
Division of the NASA Science Mission Directorate, the National Science
Foundation under Grant No.~AST-1238877, the University of Maryland, and Eotvos
Lorand University (ELTE). Some of the data presented herein were obtained at the
W.M.~Keck Observatory, which is operated as a scientific partnership among the
California Institute of Technology, the University of California and the
National Aeronautics and Space Administration. The Observatory was made possible
by the generous financial support of the W.M.~Keck Foundation. The authors wish
to recognize and acknowledge the very significant cultural role and reverence
that the summit of Mauna Kea has always had within the indigenous Hawaiian
community. We are most fortunate to have the opportunity to conduct observations
from this mountain. Observations reported here were obtained at the MMT
Observatory, a joint facility of the Smithsonian Institution and the University
of Arizona.

{\it Facilities:} \facility{PS1}, \facility{Keck:I (DEIMOS)}, \facility{MMT (Hectochelle)}

\bibliographystyle{apj}
\bibliography{ms}

\clearpage

\begin{deluxetable*}{lrrrrrrrrrrr}
\tabletypesize{\scriptsize}
\setlength{\tabcolsep}{0.02in}
\tablecolumns{12}
\tablewidth{0pc}
\tablecaption{Ophiuchus Stream Member Stars\label{table1}}
\tablehead{
\colhead{Name} & \colhead{R.A.} & \colhead{Decl.} &
\colhead{$g_{P1}$} & \colhead{$r_{P1}$} & \colhead{$i_{P1}$} &
\colhead{$z_{P1}$} & \colhead{$y_{P1}$} &
\colhead{$v_{los}$} & \colhead{DM} &
\colhead{$\mu_l$} & \colhead{$\mu_b$} \\
\colhead{ } & \colhead{(deg)} & \colhead{(deg)} &
\colhead{(mag)} & \colhead{(mag)} & \colhead{(mag)} &
\colhead{(mag)} & \colhead{(mag)} &
\colhead{(km s$^{-1}$)} & \colhead{(mag)} &
\colhead{(mas yr$^{-1}$)} & \colhead{(mas yr$^{-1}$)}
}
\startdata
bhb1 & 241.52271 & -7.01555 & $16.05\pm0.02$ & $16.11\pm0.02$ & $16.22\pm0.01$ & $16.25\pm0.02$ & $16.25\pm0.02$ & $286.7\pm1.8$ & $14.84_{-0.02}^{+0.03}$ & $-2.4\pm2.0$ & $0.5\pm1.5$ \\
bhb2 & 241.49994 & -7.03409 & $16.02\pm0.02$ & $16.09\pm0.02$ & $16.21\pm0.02$ & $16.25\pm0.02$ & $16.23\pm0.02$ & $285.3\pm1.9$ & $14.85_{-0.03}^{+0.03}$ & $-2.4\pm1.8$ & $-0.1\pm1.6$ \\
bhb3 & 242.13551 & -6.87785 & $15.96\pm0.02$ & $15.99\pm0.01$ & $16.11\pm0.02$ & $16.16\pm0.02$ & $16.13\pm0.02$ & $290.0\pm1.8$ & $14.71_{-0.02}^{+0.03}$ & $-7.4\pm2.0$ & $4.0\pm1.6$ \\
bhb4 & 241.94714 & -6.88995 & $16.22\pm0.02$ & $16.32\pm0.01$ & $16.48\pm0.02$ & $16.54\pm0.02$ & $16.56\pm0.02$ & $291.3\pm2.2$ & $14.74_{-0.02}^{+0.03}$ & $-4.3\pm1.9$ & $4.6\pm1.6$ \\
bhb6 & 242.33018 & -6.84405 & $15.67\pm0.01$ & $15.64\pm0.01$ & $15.59\pm0.01$ & $15.60\pm0.02$ & $15.54\pm0.02$ & $290.8\pm1.5$ & $14.67_{-0.02}^{+0.03}$ & $-2.5\pm1.8$ & $6.8\pm1.7$ \\
bhb7 & 242.91469 & -6.69329 & $15.66\pm0.01$ & $15.56\pm0.02$ & $15.57\pm0.01$ & $15.54\pm0.02$ & $15.50\pm0.02$ & $289.8\pm1.5$ & $14.55_{-0.03}^{+0.04}$ & $-5.9\pm2.0$ & $8.3\pm1.6$ \\
\hline
bhb6 & & & & & & & & $289.8\pm0.8$ & & & \\
rgb1 & 241.51689 & -6.98511 & $17.71\pm0.02$ & $17.18\pm0.02$ & $16.91\pm0.02$ & $16.78\pm0.02$ & $16.71\pm0.02$ & $286.0\pm0.8$ & $14.84_{-0.03}^{+0.03}$ & $-4.9\pm2.0$ & $-0.1\pm1.8$ \\
rgb2 & 241.94649 & -6.86113 & $17.34\pm0.02$ & $16.74\pm0.02$ & $16.46\pm0.02$ & $16.32\pm0.02$ & $16.25\pm0.02$ & $286.7\pm0.6$ & $14.74_{-0.02}^{+0.03}$ & $-6.1\pm2.1$ & $2.7\pm1.7$ \\
rgb3 & 241.96089 & -6.89873 & $17.64\pm0.02$ & $17.08\pm0.02$ & $16.83\pm0.02$ & $16.68\pm0.02$ & $16.62\pm0.02$ & $287.5\pm0.7$ & $14.74_{-0.02}^{+0.03}$ & $-6.5\pm2.0$ & $3.8\pm1.8$ \\
rgb4 & 242.26139 & -6.90190 & $17.05\pm0.02$ & $16.45\pm0.02$ & $16.17\pm0.01$ & $16.02\pm0.02$ & $15.93\pm0.02$ & $288.8\pm0.5$ & $14.70_{-0.02}^{+0.03}$ & $-30.0\pm1.8$ & $-19.1\pm1.5$ \\
rgb5 & 242.14832 & -6.79765 & $17.79\pm0.02$ & $17.23\pm0.02$ & $16.96\pm0.02$ & $16.82\pm0.02$ & $16.74\pm0.02$ & $288.0\pm0.9$ & $14.69_{-0.02}^{+0.03}$ & $-6.8\pm2.1$ & $-0.6\pm1.9$ \\
sgb1 & 241.95962 & -6.86881 & $18.90\pm0.02$ & $18.53\pm0.02$ & $18.38\pm0.02$ & $18.31\pm0.02$ & $18.28\pm0.02$ & $289.4\pm2.2$ & $14.74_{-0.02}^{+0.03}$ & $-6.5\pm3.0$ & $-0.2\pm2.3$ \\
msto1 & 242.02040 & -6.84122 & $19.22\pm0.02$ & $18.89\pm0.02$ & $18.74\pm0.02$ & $18.69\pm0.02$ & $18.62\pm0.03$ & $291.8\pm2.2$ & $14.72_{-0.02}^{+0.03}$ & $-9.3\pm3.2$ & $6.8\pm2.8$ \\
msto2 & 242.18360 & -6.84056 & $19.10\pm0.02$ & $18.70\pm0.02$ & $18.54\pm0.02$ & $18.45\pm0.02$ & $18.42\pm0.03$ & $286.4\pm2.6$ & $14.70_{-0.02}^{+0.03}$ & $-6.1\pm3.0$ & $3.7\pm2.5$
\enddata
\tablecomments{The horizontal line separates stars observed by DEIMOS and
Hectochelle. The name indicates the likely evolutionary stage inferred from
isochrone fitting. The PS1 photometry is {\em not} corrected for extinction. The
uncertainty in $v_{los}$ includes the uncertainty from cross-correlation/fitting
and the uncertainty in the zero-point of wavelength calibration. The DM
indicates the average DM of the stream at the position of the star, and
the uncertainties are $68\%$ confidence limits.}
\end{deluxetable*}

\end{document}

To explore correlations between distance modulus, age, and mass-loss
parameter $\eta$, we plot their joint posterior ditributions in
Figure~\ref{DM_plots}. We find that DM does not correlate with the mass-loss
parameter $\eta$, and only weakly correlates with age. However, age and $\eta$
are strongly anti-correlated. {\bf Brani: I think we can live without this
paragraph and Figure~\ref{DM_plots}.}

\begin{figure}
\plotone{DM_plots.pdf}
\caption{
The top left panel shows the marginal posterior distribution of distance modulus
of the Ophiuchus stream at $l=5\arcdeg$ (parameter $DM$). The remaining panels
show joint posterior distributions of age, distance modulus and mass-loss
parameter $\eta$ (grayscale pixels). The $68\%$ ($1\sigma$) and $95\%$
($2\sigma$) CL contours are shown as solid and dashed lines, respectively.
\label{DM_plots}}
\end{figure}
